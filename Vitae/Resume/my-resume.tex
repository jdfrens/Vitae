% resume.tex
%
% (c) 2002 Matthew Boedicker <mboedick@mboedick.org> (original author) http://mboedick.org
% (c) 2003-2007 David J. Grant <davidgrant-at-gmail.com> http://www.davidgrant.ca
% (c) 2008 Jeremy D. Frens <jdfrens@gmail.com> http://www.norecess.org/
%
%This work is licensed under the Creative Commons Attribution-Noncommercial-Share Alike 2.5 License. To view a copy of this license, visit http://creativecommons.org/licenses/by-nc-sa/2.5/ or send a letter to Creative Commons, 543 Howard Street, 5th Floor, San Francisco, California, 94105, USA.
\documentclass[letterpaper,10pt]{article}

%-----------------------------------------------------------
%Margin setup
\setlength{\voffset}{0.1in}
\setlength{\paperwidth}{8.5in}
\setlength{\paperheight}{11in}
\setlength{\headheight}{0in}
\setlength{\headsep}{0in}
\setlength{\textheight}{11in}
\setlength{\textheight}{9.5in}
\setlength{\topmargin}{-0.25in}
\setlength{\textwidth}{7in}
\setlength{\topskip}{0in}
\setlength{\oddsidemargin}{-0.25in}
\setlength{\evensidemargin}{-0.25in}

%-----------------------------------------------------------
%\usepackage{fullpage}
\usepackage{shading}
\usepackage{hyperref}

%\textheight=9.0in
\pagestyle{empty} 
\raggedbottom 
\raggedright

\setlength{\tabcolsep}{0in}

%-----------------------------------------------------------
%Custom commands
\newcommand{\resitem}[1]{\item #1 \vspace{-2pt}}

\newcommand{\resheading}[1]{{\large \parashade[.9]{sharpcorners}{\textbf{#1 \vphantom{p\^{E}}}}}} 

\newcommand{\ressubheading}[4]{ 
\begin{tabular*}{7in}{l@{\extracolsep{\fill}}r}
	\textbf{#1} & #2 \\
	{#3} & {#4} \\
\end{tabular*}
\vspace{-6pt}}

\newcommand{\me}[1]{\emph{#1}}

\newcommand{\duphref}[1]{\href{#1}{#1}}
%-----------------------------------------------------------
\newenvironment{comment}{
\begin{quote}\it
}{
\end{quote}
}
\newenvironment{project}[4]{
\setlength{\parskip}{0pt}
\setlength{\topsep}{0pt}
\setlength{\itemsep}{0pt}
\setlength{\parsep}{0pt}
\setlength{\partopsep}{0pt}
\item \ressubheading{#1}{#2}{#3}{#4}
}{\medskip}

\renewcommand{\labelitemi}{}
\setlength{\leftmargini}{0mm}

%=============================================================================
%=============================================================================


\begin{document} 
\begin{tabular*}
	{7in}{l@{\extracolsep{\fill}}r} \textbf{\Huge Jeremy D.~Frens, Ph.D.} & 616-308-3979\\
	2720 Harbor Dr. \#304 & jdfrens@gmail.com \\
	Grand Rapids, MI 49512 & http://www.norecess.org/\\
\end{tabular*}
\\

\vspace{0.1in}



%=============================================================================

\resheading{Summary}

Software developer trapped in body of academic, looking to break free.

{
\setlength{\leftmargini}{10mm}
\setlength{\itemsep}{0mm}
\setlength{\parsep}{0mm}
\begin{itemize}
\renewcommand{\labelitemi}{$\bullet$}
	\item Project leader on several open source projects.
	\item Agile developer with excellent experience with test-driven development.
	\item Programmer in Java, Ruby, \LaTeX, C/C++, Scheme, Haskell, and others.
	\item System administrator on Linux (Gentoo, Ubuntu, SuSE) and on Mac~OS~X systems.
	\item Teacher, over 15 years experience at the college level.
	\item Writer of academic research as well as laboratory manuals for programming courses.
\end{itemize}
}



%=============================================================================
% \resheading{Leadership Experience}
% 
% \begin{itemize}
% 
% 	\item Project Leadership 
% 	\begin{itemize}
% 		\item Teaching as an assistant during graduate school and now as a professor. 
% 		\item Mentoring students in senior projects at Calvin involving large software products.
% 		\item Project lead on YAGS, a Ruby-on-Rails project using agile development.
% 		\item Leading own open source projects: CITkit, CIAT, No Latte, and ANTLR Testing.
% 	\end{itemize}
% 
% 	\item Written Leadership
% 	  \begin{itemize}
% 	  	\item Wrote reference material for a website administration course.
% 	 		\item Retooled laboratory exercises for introductory programming course in C++ (\textit{Hands on C++}).
% 	 		\item Retooled and rewrote laboratory exercises for introductory programming course in Java, adding JUnit as a primary development tool (\textit{Hands on Testing Java}).
% 			\item Developed and wrote laboratory exercises for programming-languages course to develop interpreters iteratively using test-driven development (\textit{Incremental Development of Interpreters}).
% 			\item Wrote documentation for open source projects.
% 	  \end{itemize}
% 
% \end{itemize}


%=============================================================================
\resheading{Significant Projects} 
\begin{itemize}

	\begin{project}{YAGS}{Calvin College}{Project Lead and Programmer}{Summer 2007}
		\begin{comment}
			YAGS (Yet Another Genetics Simulator) is a Ruby-on-Rails webapp that simulates Mendelian genetics to teach genetics to college-level biology students.
		\end{comment}
		\begin{itemize}
			\resitem{Managed two student programmers.}
			\resitem{Used Extreme Programming.}
				\begin{itemize}
					\resitem{Weekly planning meetings with on-site client.}
					\resitem{Test-driven development.}
					\resitem{Continuous integration with \href{http://cruisecontrolrb.thoughtworks.com/}{CruiseControl.rb}}
					\resitem{Week-long iterations.}
				\end{itemize}
			\resitem{Developed and reviewed code; maintained subversion repository and CruiseControl.rb server.}
		\end{itemize}
	\end{project}

	\begin{project}{No Latte}{SourceForge}{Sole Programmer}{Summer 2003--present}
	  \begin{comment}
	  	\href{http://nolatte.sourceforge.net/}{No Latte} is an interpreter for a variation of the \href{http://www.latte.org}{Latte} language for writing XHTML documents in a functional-programming style---\LaTeX\ sensibilities with LISP semantics.
	  \end{comment}
		\begin{itemize}
			\resitem{Written in Java.}
			\resitem{Uses test-driven development with JUnit, EasyMock, and FitNesse.}
			\resitem{Uses \href{http://www.antlr.org/}{ANTLR} for the front-end.}
		\end{itemize}
	\end{project}
	
	\begin{project}{ANTLR Testing}{SourceForge}{Sole Programmer}{Summer 2003--present}
		\begin{comment}
			\href{http://antlr-testing.sourceforge.net/}{ANTLR Testing} is a unit-testing library for \href{http://www.antlr.org/}{ANTLR} grammars based on \href{http://junit.org/}{JUnit}.
		\end{comment}
		\begin{itemize}
			\resitem{Written in Java.}
			\resitem{Used in No Latte project as well as in programming-language course.}
		\end{itemize}
	\end{project}
	
	\begin{project}{Department Website}{Calvin College}{Sole Programmer}{January 2007--present}
	  \begin{comment}
	  	A modest CMS for a department website at Calvin College.
	  \end{comment}
	  \begin{itemize}
	  	\resitem{Written as a Ruby-on-Rails webapp.}
			\resitem{Fully test driven.}
	  \end{itemize}
	\end{project}

	\begin{project}{CIAT}{GitHub}{Sole Programmer}{Summer 2008--present}
	  \begin{comment}
	  	\href{https://github.com/jdfrens/ciat/tree}{CIAT} is a testing framework for writing acceptance tests for interpreters and compilers.
	  \end{comment}
	  \begin{itemize}
	  	\resitem{Written as a Ruby gem.}
			\resitem{To be used in a programming-languages course, Spring~2009.}
	  \end{itemize}
	\end{project}

\end{itemize}


\resheading{Work Experience in Academia}

% TODO: courses taught
% TODO: professional organizations
\begin{itemize}
	\item \ressubheading{Assistant Professor}{Calvin College}{Grand Rapids, MI}{2000--present} 
	  \begin{itemize}
		\resitem{Taught a variety of courses: introductory programming in C++ and Java, website administration, programming languages, automata and grammars, compilers.}
			% 		  \begin{itemize}
			% 		  \item Introductory programming in C++ and in Java.
			% \item Technical courses in office tools (e.g., Excel, Dreamweaver, PowerPoint).
			% \item Website administration.
			% \item Programming languages; compilers.
			% 		  \end{itemize}
		\resitem{Added to the curriculum.}
		\begin{itemize}
			\item Introduced unit testing in introductory programming course.
			\item Added many agile techniques to programming-languages course.
		\end{itemize}
		\resitem{Typical professor duty: student advising, department and college committees.}
		\resitem{Served as advisor to computer-science student club; awarded ``Outstanding Advisor'' award in 2004.}
	\end{itemize}
	
	\item \ressubheading{Java Instructor}{Rapistan/Seimens}{Grand Rapids, MI}{????}
	\begin{itemize}
		\resitem Taught Java and object-oriented programming to VisualBasic developers.
	\end{itemize}
	
	\item \ressubheading{Assistant Professor}{Northwestern College}{Orange City, IA}{1998--2000} 
	\begin{itemize}
		\resitem{Taught mostly upper-level courses.}
		  % \begin{itemize}
		  % 				\item Data structures.
		  % 	\item Programming languages.
		  % 				\item Computer architecture.
		  % 				\item Ray tracing.
		  % \end{itemize}
		\resitem{Served on department and college committees.}
	\end{itemize}
	
	\item \ressubheading{Associate Instructor}{Indiana University}{Bloomington, IN}{1992--1998} 
	\begin{itemize}
		\resitem{Assisted and graded various courses: introductory programming, programming languages, data structures.}
		\resitem{Taught courses in summer as primary instructor: introductory programming, data structures.}
		\resitem{Awarded ``Outstanding Associate Instruction'' from Computer Science Department in 1998.}
	\end{itemize}
\end{itemize}


\resheading{Education} 
\begin{itemize}
	\item \ressubheading{Indiana University}{Bloomington, IN}{Ph.D., Computer Science}{2002} 
	\begin{itemize}
		\resitem{Specialized in Functional Programming, Programming Languages, and Scientific Computing.}
		\resitem{Examined the memory and parallelism benefits of a block-recursive decomposition of matrix structures and algorithms.}
			% 		  \begin{itemize}
			% \item Developed algorithms for matrix-matrix multiplication and $QR$ factorization.
			% 		  \item Wrote C code in a functional style.
			% 		  \item Achieved scalable performance without having to retune algorithms for each new architecture.
			% 		  \end{itemize}
	\end{itemize}
	
	\item \ressubheading{Indiana University}{Bloomington, IN}{M.S., Computer Science}{1994}	
	   \begin{itemize}
	   \resitem{Important courses: Programming Languages, Compilers (2~semesters), Computer Graphics (2~semesters)}
	   \end{itemize}
	
	\item \ressubheading{Calvin College}{Grand Rapids, MI}{B.A., Computer Science and Mathematics}{1992} 
  	\begin{itemize}
	  \resitem{Important courses: Real Analysis (2~semesters), Linear Algebra, Abstract Algebra, Advanced Logic, Topology, Compilers, Databases, Operating Systems, Programming Languages}
		\resitem{Awarded the Rinck Prize in mathematics, 1992} 
	  \end{itemize}
\end{itemize}

\resheading{Publications and Presentations}

A complete curriculum vitae is available at \duphref{http://norecess.org/jeremy-d-frens-vitae.pdf}.  Copies and access to publications available upon request.

\begin{itemize}
	\item \href{http://cs.calvin.edu/curriculum/cs/214/jdfrens/Labs/}{\textit{Incremental Development of Interpreters}}.  In progress, used in Programming Languages Course (Spring 2008, Spring 2009).
	\begin{quote}
		Develops interpreters incrementally using test-driven development.  Technologies include Java, JUnit, ANTLR, ANTLR Testing, and FitNesse (or CIAT).
	\end{quote}
	  % \begin{itemize}
	  % 	\item Develops interpreters incrementally using test-driven development.
	  % 			\item Uses Java, JUnit, ANTLR, ANTLR Testing, and FitNesse.
	  % \end{itemize}
	
	% TODO: link to slides?
	\item \textit{Ruby and Rails}.  Invited talk at monthly meeting of AITP West Michigan, 21 February 2008.
	
	\item \duphref{http://doi.acm.org/10.1145/1121341.1121372}{``15 Compilers in 15 Days''} with Andy~Meneely (student).  \textit{Proceedings of the 2006 ACM Symposium on Computer Science Education} (2006 March), 92--95.
	  \begin{itemize}
		\item Describes success at developing compilers incrementally.
		\item Explains pedagogical benefits of incremental development.
		\item Used to motivate \textit{Incremental Development of Interpreters}.
  	\end{itemize}
	
	\item \href{http://cs.calvin.edu/books/c++/intro/3e/HandsOnC++/}{\textit{Hands on C++} (2003), 3e}, with Joel C.\ Adams.  Prentice Hall.
	  \begin{itemize}
	  	\item Significant re-write of 2nd edition.
	  \end{itemize}
	  
	\item \href{http://cs.calvin.edu/curriculum/cs/108/HoTJ/}{\textit{Hands on Testing Java}} based on material by Joel C.\ Adams and Charles Hoot.
	  \begin{itemize}
	  	\item Significant re-write of original \textit{Hands on Java}.
			\item Improved the handling of object-oriented programming.
			\item Added unit-testing with JUnit to every lab.
	  \end{itemize}
  
\end{itemize}


\end{document} 
