\documentclass{article}

\usepackage{fullpage}
\usepackage{hyperref}

\newcommand{\duphref}[1]{\href{#1}{#1}}
\newcommand{\term}[1]{\textbf{#1}}

\title{Five-Year Plan for Scholarship}
\author{Jeremy D.\ Frens}
\date{Spring 2005}

\begin{document}

\maketitle

\section*{Purpose}

There is a software crisis (of sorts) today.  Programmers are traditionally and habitually overworked with impossible and imprecise expectations; software is routinely released late and buggy.  Agile software development is a modern attempt to address these issues.

\term{Agile software development} favors people over process, product, and documentation.  So programmers are not overworked, specifications are precise, and software is delivered on time.  Agile software development also favors writing simple, tested code.

My main interest is in \term{test-driven development} (TDD), a key component to just about all agile development techniques.  Using TDD, a programmer first writes a test in code for a new feature, and this test will fail; the code is \term{refactored} (textually changed without changing behavior) so that new code can be added; then new code is added for the new feature.  The code is refactored and modified until all of the tests, new and old, pass.  This process is known as \term{unit testing} the code.  By working on the code incrementally like this, the software system will grow into the final product, being tested at each step along the way.

I have found this process to be amazingly effective in my own software projects; I am interested in how TDD is effective and how it can be made more effective.  I have most recently started to think about using TDD development in the courses I teach.  I am already using unit testing in our introductory programming course; the students use it as a basic tool to make sure that their computation code works correctly.  Inspired by that success, I have thought about how to work all of TDD into the curriculum.  Introductory students do not have the skill set to do any real refactoring, so full TDD is beyond their reach.  After three or more semesters of programming, though, some simple refactoring is well within a student's reach.  I have recently come to realize that TDD may have certain advantages pedegogically.  At each stage of development, students will have a working system, front to back; as each new feature is added, different parts of the system will be stressed in different ways.  Perhaps one week is spent changing the input library a lot; the following week, some intermediate computations are modified; but each week, the student has a complete working system.

\section*{My Research}

Completed work:
\begin{itemize}
\item \textit{Hands on Testing Java}, a lab manual for introductory Java programming using JUnit, a unit-testing library.
\item ``Fifteen Compilers in Fifteen Days'', a prototype interim course done Spring 2005 with Andy Meneely.  Pending: finish compiler, SIGCSE 2006 paper.
\end{itemize}

Future work:
\begin{itemize}
\item ``Fifteen Raytracers in Fifteen Days'', a prototype interim course done Spring 2006 (perhaps with Meneely and/or others).  SIGCSE 2007 paper, OOPSLA 2006 paper.
\item Incremental, TDD programming languages.  Rework labs of Programming Languages course to implement one feature per week.  See if lab manual is interesting to publishers.
\item CCEL Desktop.  Use TDD (and other agile development practices) to work on CCEL desktop software.
\end{itemize}

Agile software development and TDD are not embraced by all practitioners or educators.  My goal is to show that TDD can be used to teach upper-level courses (as well as provide an outlet for students to practice TDD); in fact, I hope to show that there are distinct pedegogical advantages of using a TDD approach, despite the overhead of teaching TDD itself.

\section*{Bibliography}

\begin{itemize}
\item Kent Beck, \textit{Extreme Programming Explained}.
\item Martin Fowler, \textit{Refactoring}.
\item David Astels, \textit{Test-Driven Development}
\end{itemize}

\section*{Skills and Resources}

\begin{itemize}
\item It would be good to see how real unit tests are written in real projects.
\item Am I using all of the useful tools I can?
\item Student guinea pigs (like Meneely).
\item Resource: CVS server.
\end{itemize}

\end{document}
