\documentclass{article}
\pagestyle{empty}

\usepackage{fullpage}
\usepackage{hyperref}

\newcommand{\duphref}[1]{\href{#1}{#1}}
\newcommand{\term}[1]{\textbf{#1}}
\newcommand{\person}[1]{\textbf{#1}}

\title{Benefits of My Externship}
\author{Jeremy D.\ Frens}
\date{Fall 2006}

\begin{document}

\maketitle

\thispagestyle{empty}

\section*{Benefits to Me}

My research interests are in Programming Languages (PL), the study of how to best express computer programs.  One aspect of PL that has greatly interested me is agile software development (which intersects quite a lot with Software Engineering, another area of Computer Science).  I am interested in agile software development in three different ways:
  \begin{itemize}
  \item in terms of its theoretical foundations,
  \item as a pratical way to write sofware, and
  \item as an instructional technique.
  \end{itemize}
The last interest has some great research possibilities.  I am presenting a paper in March on this topic, and my sabbatical next spring is based on this interest.

However, my experience with agile software development has been mostly theoretical and self-taught.  An externship at Atomic Object will give me a practical view of agile software development which I can learn from actual practitioners.  So the externship will balance out my knowledge of agile software development.  This balanced education will strengthen both my theoretical research and my teaching in the obvious ways.

The externship will also strengthen me as an advisor.  I am a good advisor for students who intend to go to graudate school since that is what I am most familiar with; I am an adequate advisor for students who go directly into the business world after graduation since I have no practical experience in getting a job in computing.  I would like to be a good (or great) advisor for both sets of students, and this externship can and will play an important role in acheiving this goal.

\section*{Benefits to the College}

Agile software development is playing a larger and larger role in Computer Science these days, and few of us in the CS Department have training in this area.  While it is not necessary for all of us to be agile developers, it is appropriate that several of us do have some knowledge and experience in this area.  We need to be able to teach it to our students; it can also work as a recruiting tool.

Also, if I am correct that agile software development may hold some answers to better teaching of computer science, then it will make our department stronger to take advantage of this improved teaching.

%% TODO: benefits to the firm

\end{document}

