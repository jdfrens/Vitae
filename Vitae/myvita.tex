\documentclass[ComputerScience]{vita}

\usepackage{hyperref}

\newcommand{\duphref}[1]{\href{#1}{#1}}
\newcategory[Software Projects]{software}
\newcategory[Lab Manuals]{labmanuals}

\begin{document}

\name{Jeremy D.\ Frens}

\businessAddress{Calvin College \\
Computer Science Department \\
1740 Knollcrest Circle SE              \\
Grand Rapids, MI 49546-4403      \\
616-526-8666                \\
jdfrens@calvin.edu          \\
}

\homeAddress{2720 Harbor Dr.~\#304      \\
Kentwood, MI 49512         \\
616-308-3979               \\
jdfrens@gmail.com             \\
}




\begin{vita}



\begin{Degrees}
\item Ph.D.\ in Computer Science (2002), Indiana University.
%   Dissertation: \textit{Matrix Factorization Using a Block-Recursive
%     Structure and Block-Recursive Algorithms}
\item M.S.\ in Computer Science (1994), Indiana University.
\item B.A.\ in Computer Science and Mathematics (1992), Calvin College.
\end{Degrees}




\begin{Experience}

\item \emph{Since Fall 2000.}  Assistant Professor of Computer Science at Calvin College.
  \begin{itemize}
  \item \emph{Fall 2006.}  Externship at Atomic Object.
  \end{itemize}

\item \emph{Fall 1998 through Fall 2000.}  Assistant Professor of Computer Science at Northwestern College (Orange City, IA).

\item \emph{Fall 1992 through Fall 1998.}  Associate Instructor at Indiana University.  Includes three summers of teaching as an Instructor, with full control of a course.

% \item \emph{Summer 1994.}  Developed course material for an independent study ``Introduction to Computing'' course at Indiana University.

\end{Experience}




\begin{Honors}

\item Outstanding Associate Instructor (1998) from the Computer Department at Indiana University.

\item Rinck Memorial Prize (1992) from the Math and Computer Department at Calvin College.

\end{Honors}


\newpage



\begin{Publications}

  \item \textit{Matrix Factorization Using a Block-Recursive Structure and Block-Recursive Algorithms}.  Ph.D.\ Dissertation (2002).  Available as Technical Report 568, Computer Science Department, Indiana University. \\\duphref{http://www.cs.indiana.edu/Research/techreports/TR568.shtml}

  \begin{Papers at Refereed Conferences}
	
  \item 15 Compilers in 15 Days.  \textit{Proc.\ 2006 ACM Symp.\ on Computer Science Education} (2006 March), 92--95.  With A.~Meneely. \\\duphref{http://doi.acm.org/10.1145/1121341.1121372}

  \item Taming the tiger: teaching the next version of Java.  \textit{Proc.\ 2004 ACM Symp.\ on Computer Science Education} (2004 March), 151--155.  \\\duphref{http://doi.acm.org/10.1145/971300.971356}

  \item Factorization with Morton-ordered quadtree matrices for memory re-use and parallelism. \textit{Proc.\ 2003 ACM Symp.\ on Principles and Practice of Parallel Programming} (2003 June), 144--154.  With D.S.\ Wise.  \\\duphref{http://doi.acm.org/10.1145/781498.781525}

  \item Object centered design for Java: teaching OOD in CS-1. \textit{Proc. 2003 ACM Symp.\ on Computer Science Education} (2003 February), 273--277.  With J.\ Adams.  \\\duphref{http://doi.acm.org/10.1145/611892.611986}

  \item Language support for Morton-order matrices. \textit{Proc. 2001 ACM Symp.\ on Principles and Practice of Parallel Programming, SIGPLAN Not.} 36, 7 (2001 July), 24--33.  With D.S.\ Wise, Y.\ Gu, and G.A.\ Alexander.  \\\duphref{http://doi.acm.org/10.1145/379539.379559}

  \item Auto-blocking matrix-multiplication, or Tracking BLAS3 performance from source code. \textit{Proc. 1997 ACM Symp.\ on Principles and Practice of Parallel Programming\nocorr}, \textit{ACM SIGPLAN Notices} \textbf{32}, 7, (July 1997) 206-216.  With D.S.~Wise. \\\duphref{http://doi.acm.org/10.1145/263764.263789}

  \end{Papers at Refereed Conferences}

  \begin{Technical Reports}

  \item Morton-order Matrices Deserve Compilers' Support. Technical Report 533, Computer Science Department, Indiana University (November 1999).  With D.S.~Wise.  \\\duphref{http://www.cs.indiana.edu/Research/techreports/TR533.shtml}

  \item Matrix inversion using quadtrees implemented in Gofer.  Technical Report 433, Computer Science Department, Indiana University (May 1995).  With D.S.~Wise.  \\\duphref{http://www.cs.indiana.edu/Research/techreports/TR433.shtml}

  \end{Technical Reports}

  \begin{labmanuals}
  
  \item \textit{Agile Development of Interpreters}, a lab manual for a Programming-Language course.  In progress.  \\\duphref{http://cs.calvin.edu/curriculum/cs/214/jdfrens/Labs/}.
  
  % TODO: URL goes to a brain-dead webpage
  \item \textit{Hands on Testing Java}, a lab manual for a first-semester programming course using Java and JUnit.  In progress.  \\\duphref{http://cs.calvin.edu/curriculum/cs/108/HoTJ/}.

  \item \textit{Hands on C++} (2003), 3e.  Prentice Hall.  With J.\ Adams.  \\\duphref{http://cs.calvin.edu/books/c++/intro/3e/HandsOnC++/}

  \end{labmanuals}

\end{Publications}

\begin{software}
  \item YAGS (Yet Another Genetics Simulator).  Project lead with two student developers.  Used Ruby on Rails to simulate the genetics of fruit flies for genetic experiments in Biology courses.
  
  \item No Latte (\duphref{http://nolatte.sourceforge.net/}).  Sole developer on a No Latte interpreter.  No Latte is a language for writing web pages; interpreter is written in Java using JUnit, FitNesse, and ANTLR.
  
  \item ANTLR Testing (\duphref{http://antlr-testing.sourceforge.net/}).  Sole developer on a extension to JUnit for testing lexers and parsers generated by ANTLR.

%  \item A project lead on CCEL Desktop application.  The CCEL Desktop provides books from CCEL on a user's local machine complete with searching capabilities.  \\\duphref{http://ccel-desktop.sourceforge.net/}
\end{software}

\begin{Memberships}

\item \href{http://www.acm.org/}{Association for Computing Machinery} (ACM).  Includes membership in special interest groups for computers and society, computer science education, programming languages, and computer graphics.

% \item IEEE Computer Society (affiliate member).

\item Sigma Xi.

\item User groups:
  \begin{itemize}
  \item \href{http://www.xpwestmichigan.org/}{XP West Michigan} 
  \item \href{http://www.gr-jug.org/}{Grand Rapids Java User Group}
  \item \href{http://gr-ruby.org/}{Michigan Ruby User Group}
  \end{itemize}

\end{Memberships}

\begin{References}

David S. Wise.  Professor of Computer Science, Indiana University (Bloomington).

% % Dennis Gannon.  Professor and Chair of Computer Science, Indiana
% % University (Bloomington).

% % George Springer.  Professor Emeritus of Computer Science, Indiana
% % University (Bloomington).

% % David A. Bremer.  Pastor of United Presbyterian Church, Bloomington,
% % IN.

\end{References}

% TODO: technical skills

\end{vita}
\end{document}
