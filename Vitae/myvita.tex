\documentclass[ComputerScience]{vita}

\usepackage{hyperref}

\newcommand{\duphref}[1]{\href{#1}{#1}}
\newcategory[Software Projects]{software}

\begin{document}

\name{Jeremy D.\ Frens}

\businessAddress{Calvin College \\
Computer Science Department \\
1740 Knollcrest Circle SE              \\
Grand Rapids, MI 49546-4403      \\
616-526-8666                \\
jdfrens@calvin.edu          \\
}

\homeAddress{2720 Harbor Dr.~\#304      \\
Kentwood, MI 49512         \\
616-308-3979               \\
jdfrens@acm.org             \\
}




\begin{vita}



\begin{Degrees}
\item Ph.D.\ in Computer Science (2002), Indiana University.
%   Dissertation: \textit{Matrix Factorization Using a Block-Recursive
%     Structure and Block-Recursive Algorithms}
\item M.S.\ in Computer Science (1994), Indiana University.
\item B.A.\ in Computer Science and Mathematics (1992), Calvin College.
\end{Degrees}




\begin{Experience}

\item \emph{Since Fall 2000.}  Assistant Professor of Computer
  Science at Calvin College.

\item \emph{Fall 1998 through Fall 2000.}  Assistant Professor of
  Computer Science at Northwestern College (Orange City, IA).

\item \emph{Fall 1992 through Fall 1998.}  Associate Instructor at
  Indiana University.  Includes three summers of teaching as an
  Instructor, with full control of a course.

\item \emph{Summer 1994.}  Developed course material for an
  independent study ``Introduction to Computing'' course at Indiana
  University.

\end{Experience}




\begin{Honors}

\item Rinck Memorial Prize (1992) from the Math and Computer
  Department at Calvin College.

\item Outstanding Associate Instructor (1998) from the Computer
  Department at Indiana University.

\end{Honors}


\newpage



\begin{Publications}

  \item \textit{Matrix Factorization Using a Block-Recursive Structure and Block-Recursive Algorithms}.  Ph.D.\ Dissertation (2002).  Available as Technical Report 568, Computer Science Department, Indiana University.  To be worked into a journal article.  \\\duphref{http://www.cs.indiana.edu/Research/techreports/TR568.shtml}

  \begin{Papers at Refereed Conferences}
	
  % TODO: It'll be published soon!
	\item 15 Compilers in 15 Days.  \textit{Proc.\ 2006 ACM Symp.\ on Computer Science Education} (2006 March), not yet published.

	\item Taming the tiger: teaching the next version of Java.  \textit{Proc.\ 2004 ACM Symp.\ on Computer Science Education} (2004 March), 151--155.  \\\duphref{http://doi.acm.org/10.1145/971300.971356}

  \item Factorization with Morton-ordered quadtree matrices for memory re-use and parallelism. \textit{Proc.\ 2003 ACM Symp.\ on Principles and Practice of Parallel Programming} (2003 June), 144--154.  With D.S.\ Wise.  \\\duphref{http://doi.acm.org/10.1145/781498.781525}

  \item Object centered design for Java: teaching OOD in CS-1. \textit{Proc. 2003 ACM Symp.\ on Computer Science Education} (2003 February), 273--277.  With J.\ Adams.  \\\duphref{http://doi.acm.org/10.1145/611892.611986}

  \item Language support for Morton-order matrices. \textit{Proc. 2001 ACM Symp.\ on Principles and Practice of Parallel Programming, SIGPLAN Not.} 36, 7 (2001 July), 24--33.  With D.S.\ Wise, Y.\ Gu, and G.A.\ Alexander.  \\\duphref{http://doi.acm.org/10.1145/379539.379559}

  \item Auto-blocking matrix-multiplication, or Tracking BLAS3 performance from source code. \textit{Proc. 1997 ACM Symp.\ on Principles and Practice of Parallel Programming\nocorr}, \textit{ACM SIGPLAN Notices} \textbf{32}, 7, (July 1997) 206-216.  With D.S.~Wise. \\\duphref{http://doi.acm.org/10.1145/263764.263789}

  \end{Papers at Refereed Conferences}

  \begin{Technical Reports}

  \item Morton-order Matrices Deserve Compilers' Support. Technical Report 533, Computer Science Department, Indiana University (November 1999).  With D.S.~Wise.  \\\duphref{http://www.cs.indiana.edu/Research/techreports/TR533.shtml}

  \item Matrix inversion Using quadtrees implemented in Gofer.  Technical Report 433, Computer Science Department, Indiana University (May 1995).  With D.S.~Wise.  \\\duphref{http://www.cs.indiana.edu/Research/techreports/TR433.shtml}

  \end{Technical Reports}

  \begin{Books}
  
  \item \textit{Hands on Testing Java} In progress.  \\\duphref{http://cs.calvin.edu/curriculum/cs/108/HoTJ/}, username \emph{ann}, password \emph{JUnit}

  \item \textit{Hands on C++} (2003), 3e.  Prentice Hall.  With J.\ Adams.  \\\duphref{http://cs.calvin.edu/books/c++/intro/3e/HandsOnC++/}

  \end{Books}

\end{Publications}

\begin{software}
	\item A project lead on CCEL Desktop application.  The CCEL Desktop provides books from CCEL on a user's local machine complete with searching capabilities.  \\\duphref{http://ccel-desktop.sourceforge.net/}
	\item The project lead on a No Latte interpreter.  No Latte is a language for writing web pages.  \\\duphref{http://nolatte.sourceforge.net/}
  \item The project lead on a extension to JUnit for testing lexers and parsers generated by ANTLR (see \duphref{http://www.antlr.org}).  \\\duphref{http://antlr-testing.sourceforge.net/}
\end{software}

\begin{Memberships}

\item Association for Computing Machinery (ACM).  Includes membership in special interest groups for computers and society, computer science education, programming languages, and computer graphics.

% \item IEEE Computer Society (affiliate member).

\item Sigma Xi.

\item XP West Michigan user group.  A group of professionals interested in Extreme Programming.

\end{Memberships}

% % \section*{References}

% % David S. Wise.  Professor of Computer Science, Indiana University
% % (Bloomington).

% % Dennis Gannon.  Professor and Chair of Computer Science, Indiana
% % University (Bloomington).

% % George Springer.  Professor Emeritus of Computer Science, Indiana
% % University (Bloomington).

% % David A. Bremer.  Pastor of United Presbyterian Church, Bloomington,
% % IN.

% \end{document}


\end{vita}
\end{document}
