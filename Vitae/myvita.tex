\documentclass[ComputerScience]{vita}

\usepackage{hyperref}

\newcommand{\duphref}[1]{\href{#1}{#1}}
\newcategory[Software Projects]{software}
\newkind{Lab Manuals}

\begin{document}

\name{Jeremy D.\ Frens}

\businessAddress{Calvin College \\
Computer Science Department \\
1740 Knollcrest Circle SE              \\
Grand Rapids, MI 49546-4403      \\
616-526-8666                \\
jdfrens@calvin.edu          \\
}

\homeAddress{2720 Harbor Dr.~\#304      \\
Kentwood, MI 49512         \\
616-308-3979               \\
jdfrens@gmail.com             \\
}

\renewcommand{\labelitemi}{}
\setlength{\leftmarginii}{\leftmargini}

\begin{vita}



\begin{Degrees}
\item Ph.D.\ in Computer Science (2002), Indiana University. \par
  {
    Dissertation: \textit{Matrix Factorization Using a Block-Recursive
    Structure and Block-Recursive Algorithms}
	}
	\item M.S.\ in Computer Science (1994), Indiana University.
  \item B.A.\ in Computer Science and Mathematics (1992), Calvin College. \par
	{
	  Awarded Rinck Memorial Prize in~1992 from the Math and Computer Department.
  }
\end{Degrees}




\begin{Experience}

\item \emph{Since Fall 2000.}  Assistant Professor of Computer Science at Calvin College. \par
  {
    \emph{Fall 2006.}  Externship at Atomic Object.
  }

\item \emph{Fall 1998 through Fall 2000.}  Assistant Professor of Computer Science at Northwestern College (Orange City, IA).

\item \emph{Fall 1992 through Fall 1998.}  Associate Instructor at Indiana University.\par
	{
		Three summers of teaching as an Instructor, with full control of a course.  Awarded Outstanding Associate Instructor in~1998.
	}

\end{Experience}

\newpage

%=============================================================================

\begin{Publications}

  \item \textit{\href{http://www.cs.indiana.edu/Research/techreports/TR568.shtml}{Matrix Factorization} Using a Block-Recursive Structure and Block-Recursive Algorithms}.  Ph.D.\ Dissertation (2002).  Available as Technical Report 568, Computer Science Department, Indiana University.

  \begin{Papers at Refereed Conferences}
	
  \item \href{http://doi.acm.org/10.1145/1121341.1121372}{15 Compilers in 15 Days}.  \textit{Proc.\ 2006 ACM Symp.\ on Computer Science Education} (2006 March), 92--95.  With A.~Meneely.

  \item \href{http://doi.acm.org/10.1145/971300.971356}{Taming the tiger: teaching the next version of Java}.  \textit{Proc.\ 2004 ACM Symp.\ on Computer Science Education} (2004 March), 151--155.  

  \item \href{http://doi.acm.org/10.1145/781498.781525}{Factorization with Morton-ordered} quadtree matrices for memory re-use and parallelism. \textit{Proc.\ 2003 ACM Symp.\ on Principles and Practice of Parallel Programming} (2003 June), 144--154.  With D.S.\ Wise.

  \item \href{http://doi.acm.org/10.1145/611892.611986}{Object centered design} for Java: teaching OOD in CS-1. \textit{Proc. 2003 ACM Symp.\ on Computer Science Education} (2003 February), 273--277.  With J.\ Adams.

  \item \href{http://doi.acm.org/10.1145/379539.379559}{Language support for Morton-order matrices}. \textit{Proc. 2001 ACM Symp.\ on Principles and Practice of Parallel Programming, SIGPLAN Not.} 36, 7 (2001 July), 24--33.  With D.S.\ Wise, Y.\ Gu, and G.A.\ Alexander.

  \item \href{http://doi.acm.org/10.1145/263764.263789}{Auto-blocking matrix-multiplication}, or Tracking BLAS3 performance from source code. \textit{Proc. 1997 ACM Symp.\ on Principles and Practice of Parallel Programming\nocorr}, \textit{ACM SIGPLAN Notices} \textbf{32}, 7, (July 1997) 206-216.  With D.S.~Wise.

  \end{Papers at Refereed Conferences}

  \begin{Technical Reports}

  \item \href{http://www.cs.indiana.edu/Research/techreports/TR533.shtml}{Morton-order Matrices Deserve Compilers' Support}. Technical Report 533, Computer Science Department, Indiana University (November 1999).  With D.S.~Wise.

  \item \href{http://www.cs.indiana.edu/Research/techreports/TR433.shtml}{Matrix inversion using quadtrees implemented in Gofer}.  Technical Report 433, Computer Science Department, Indiana University (May 1995).  With D.S.~Wise.

  \end{Technical Reports}

  \begin{Lab Manuals}
  
  \item \href{http://cs.calvin.edu/curriculum/cs/214/jdfrens/Labs/Interpreters}{\textit{Agile Development of Interpreters}}, a lab manual for a programming language course using incremental, test-driven development.  In progress.
  
  % TODO: URL goes to a brain-dead webpage
  \item \href{http://cs.calvin.edu/curriculum/cs/108/HoTJ/}{\textit{Hands on Testing Java}}, a lab manual for a first-semester programming course using Java and JUnit.  In progress.

  \item \href{http://cs.calvin.edu/books/c++/intro/3e/HandsOnC++/}{\textit{Hands on C++}} (2003), 3e.  Prentice Hall.  With J.\ Adams.

  \end{Lab Manuals}

\end{Publications}

%=============================================================================

\begin{software}
	\item \href{http://citkit.sourceforge.net/}{CITkit} (Compiler and Interpreter Toolkit).  Sole developer on Java library to assist in writing compilers and interpreters, primarily for educational purposes.
	
	\item \href{http://citkit.sourceforge.net/ciat/}{CIAT} (Compiler and Interpreter Acceptance Tester).  Developing a Ruby gem to run acceptance tests on compilers and interpreters.
	
  \item YAGS (Yet Another Genetics Simulator).  Project lead with two student developers.  Used Ruby on Rails to simulate the genetics of fruit flies for genetic experiments in Biology courses.
  
  \item \href{http://nolatte.sourceforge.net/}{No Latte}.  Sole developer on a No Latte interpreter.  No Latte is a language for writing web pages; interpreter is written in Java using JUnit, FitNesse, and ANTLR.
  
  \item \href{http://antlr-testing.sourceforge.net/}{ANTLR Testing}.  Sole developer on a extension to JUnit for testing lexers and parsers generated by ANTLR.

\end{software}

%=============================================================================

\begin{Memberships}

\item \href{http://www.acm.org/}{Association for Computing Machinery} (ACM).  Includes membership in special interest groups for computers and society, computer science education, programming languages, and computer graphics.

\item Sigma Xi.

\item User groups:
  \begin{itemize}
  \item \hspace{-\leftmargini}\href{http://www.xpwestmichigan.org/}{XP West Michigan} 
  \item \hspace{-\leftmargini}\href{http://www.gr-jug.org/}{Grand Rapids Java User Group}
  \item \hspace{-\leftmargini}\href{http://gr-ruby.org/}{Grand Rapids Ruby User Group}
  \end{itemize}

\end{Memberships}


\end{vita}
\end{document}
